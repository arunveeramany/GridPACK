\section{Generalized Slab Mapper}

The generalized slab mapper also uses functions in the generalized matrix-vector interface to build dense matrices. These matrices are dense since they are generated by taking a typical vector that corresponds to a set of variables on the buses and branches and replicating the vector for different values of the variables. An example would be a matrix formed from a time series of values for a set of variables on the buses and branches. One set of indices for the matrix corresponds to the set of variables and the other set of indices corresponds to the time series. In a certain sense, these matrices are ``fat'' vectors since instead of each variable having only one value, they have multiple values. In general, slab matrices are not square. The slab matrices are used in the Kalman filter application, but they may have applicability elsewhere.

The slab mappers use additional functions from the \texttt{\textbf{GenMatVecInterface}} in order to construct matrices. These functions are analogous to the functions for setting up vectors using the \texttt{\textbf{GenVectorMap}}. The main difference is that instead of describing a list of values, the functions describe a matrix block. The row dimension corresponds to a list of variables and the column dimension describes the number of values taken by each variable. The column dimension must be the same across all variables. The contribution to the matrix from each network component is given by the function

{
\color{red}
\begin{Verbatim}[fontseries=b]
void slabSize(int *rows, int *cols) const
\end{Verbatim}
}

The index for each row can be stored using the function

{
\color{red}
\begin{Verbatim}[fontseries=b]
void slabSetRowIndex(int irow, int idx)
\end{Verbatim}
}

This function is called by the mapper and is analogous to the \texttt{\textbf{vectorSetElementIndex}} function. For the slab matrices, there is no corresponding call for columns since the matrices are dense and all rows have the same number of non-zero columns. The indices can be retrieved by the function

{
\color{red}
\begin{Verbatim}[fontseries=b]
void slabGetRowIndices(int *idx)
\end{Verbatim}
}

which is similar to the \texttt{\textbf{vectorGetElementIndices}} function.
